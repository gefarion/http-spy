\section{Utilizando HTTPSpy}

\subsection{Instalación Debian likes}

\textbf{Repositorio actual:} \url{https://github.com/cdipietro/2012\_1c\_seginfo}	
\\
\\\indent Instalación paso a paso:
\begin{enumerate}
	\item apt-get install python python-pip python-dev python-nids
	\item pip install http\_parser
	\item cpan App::cpanminus
	\item cpanm DBI DBD::SQLite Mojolicious
	\item Enjoy!
\end{enumerate}

\subsection{Modo de Uso}

HttpSpy viene con una opción de help que indica como usar el mismo. Ingresando "python main.py -h" se obtienen las siguientes intrucciones de uso:
\\\\
	{\small
	\begin{boxedverbatim}
	usage: main.py [-h] [--pcapfile PCAPFILE | --device DEVICE] 
	[--filter FILTER]
	[--list-plugins | --help-plugin HELP_PLUGIN | --plugins PLUGINS]

A very basic http sniffer

optional arguments:
  -h, --help            show this help message and exit
  --pcapfile PCAPFILE   read a tcp stream from a file
  --device DEVICE       set the device to sniff
  --filter FILTER       set the filter (see man tcpdump)
  --list-plugins        List availables plugins
  --help-plugin HELP_PLUGIN
                        Show help for a plugin
  --plugins PLUGINS     File with configured plugins
	\end{boxedverbatim}
	}
\\\\
\textbf{Ejemplo de uso:}
\\\\
{\small
\begin{boxedverbatim}
Archivo de ejemplo de configuración de un módulo, printer.yml:
	- name      : SimplePrinter
	  delimiter : "\t"
	  format    : "shost method >Host path status_code <Content-Type"

Comando para capturar de un dispositivo:
	$python main.py --device eth0 --plugins printer.yml

Comando para procesar un pcapfile:
	$python main.py --pcapfile trafico.pcap --plugins printer.yml

Ejemplo de salida:
	192.168.0.12	GET	www.gnu.org	/	200	text/html
\end{boxedverbatim}
}

\subsection{Interfaz Web}

\subsubsection{Instalando la Aplicación Web}

Pasos de la instalación:
\\\\
{\small
\begin{boxedverbatim}
	> sudo cpan cpan App::cpanminus
	> sudo cpanm DBI
	> sudo cpanm DBD::SQLite
	> sudo cpanm YAML::Tiny
	> sudo cpanm Mojolicious
\end{boxedverbatim}
}

\subsubsection{Utilizando la Aplicación Web}

Es una pequeña aplicación web que permite realizar consultas predefinidas sobre la base de datos generada por el sniffer. 
\begin{itemize}
	\item Los informes preconfigurados se definen mediante un archivo de configuración con formato yaml.
	\item Se pueden definir informes parametrizados.
\end{itemize}

Para ponerla en ejecución hay que ejecutar la siguient sentencia desde el directorio de la web app.
\\\\
{\small
\begin{boxedverbatim}
	> morbo informes
	[Sat Aug  4 23:43:38 2012] [info] Listening at "http://*:3000".
	Server available at http://127.0.0.1:3000.
	> _
\end{boxedverbatim}
}
\\\\
Una vez puesta en ejecución podremos acceder a la misma desde la url informada 
en el resultado del comando y realizar desde ella consultas sobre los datos 
almacenados de los accesos http/https.
\\
\\Por ejemplo una de las consultas definidas es la siguiente: 
\\\\
	{\small
	\begin{boxedverbatim}
	- nombre: Listar fecha, origen, destino (like)
  query: 'SELECT date, shost, host FROM http_log WHERE host LIKE ?'
  campos: 
   - Destino
  columnas:
   - Fecha
   - Origen
   - Destino
	\end{boxedverbatim}
	}
