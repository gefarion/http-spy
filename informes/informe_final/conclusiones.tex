\section{Conclusiones}

\subsection{A modo de nota final}

A modo de conclusión podemos citar que al realizar el programa de manera 
modular nos llevo a que pudiéramos agregar más funciones de manera sencilla. 

Las librerías utilizadas para el análisis del tráfico también nos ayudaron 
a analizar de manera más eficiente el tráfico http/https de la red. Pudiéndo 
así centrarnos más en que hacer con los datos obtenidos que en como obtenerlos.

El uso de pcapfiles para algunas de las pruebas nos ayudo a probar más rápidamente 
el funcionamiento correcto de los diferentes módulos implementados para el almacenamiento
e impresión de datos, como así también para probar las consultas a la base de datos 
desde la interfaz web.

El proyecto se encuentra alojado en: \url{http://code.google.com/p/http-spy/}, la idea es continuar con el mismo hasta que pueda ser utilizado en ambientes productivos reales. Para dicho objetivo es necesario realizar pruebas en sistemas grandes y desarrollar una suite de testing bastante exhaustiva sobre el aplicativo.

En un futuro planeamos realizar diferentes plugins que permitan interactuar con sistemas IDS conocidos, así como implementar un sistema de consulta mas robusto y configurable.